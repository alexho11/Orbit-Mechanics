\documentclass[12pt
,headinclude
,headsepline
,bibtotocnumbered
]{scrartcl}
\usepackage[paper=a4paper,left=25mm,right=25mm,top=25mm,bottom=25mm]{geometry} 
\usepackage[utf8]{inputenc}
\usepackage[english]{babel}
\usepackage{fancyvrb}  % Add this line
\usepackage{graphicx}
\usepackage{multirow}
\usepackage{pdfpages}
%\usepackage{wrapfig}
\usepackage{placeins}
\usepackage{float}
\usepackage{flafter}
\usepackage{mathtools}
\usepackage{hyperref}
\usepackage{epstopdf}
\usepackage[miktex]{gnuplottex}
\usepackage[T1]{fontenc}
\usepackage{mhchem}
\usepackage{fancyhdr}
%\setlength{\mathindent}{0pt}
\usepackage{amssymb}
\usepackage[list=true, font=large, labelfont=bf, 
labelformat=brace, position=top]{subcaption}
\setlength{\parindent}{0mm}
\usepackage{listings}
\usepackage{color}

\definecolor{dkgreen}{rgb}{0,0.6,0}
\definecolor{gray}{rgb}{0.5,0.5,0.5}
\definecolor{mauve}{rgb}{0.58,0,0.82}

\lstset{ %
	language=Matlab,                % the language of the code
	basicstyle=\small\ttfamily,     % the size of the fonts that are used for the code
	numbers=left,                   % where to put the line-numbers
	numberstyle=\tiny\color{gray},  % the style that is used for the line-numbers
	stepnumber=1,                   % the step between two line-numbers. If it's 1, each line will be numbered
	numbersep=5pt,                  % how far the line-numbers are from the code
	backgroundcolor=\color{white},  % choose the background color. You must add \usepackage{color}
	showspaces=false,               % show spaces adding particular underscores
	showstringspaces=false,         % underline spaces within strings
	showtabs=false,                 % show tabs within strings adding particular underscores
	frame=single,                   % adds a frame around the code
	rulecolor=\color{black},        % if not set, the frame-color may be changed on line-breaks within not-black text (e.g. commens (green here))
	tabsize=2,                      % sets default tabsize to 2 spaces
	captionpos=b,                   % sets the caption-position to bottom
	breaklines=true,                % sets automatic line breaking
	breakatwhitespace=true,         % sets if automatic breaks should only happen at whitespace
	title=\lstname,                 % show the filename of files included with \lstinputlisting; also try caption instead of title
	keywordstyle=\color{blue},      % keyword style
	commentstyle=\color{dkgreen},   % comment style
	stringstyle=\color{mauve},      % string literal style
	escapeinside={\%*}{*)},         % if you want to add LaTeX within your code
	morekeywords={*,...}            % if you want to add more keywords to the set
}

\setlength{\parindent}{0mm}

\pagestyle{fancy}
\fancyhf{}
\lhead{Orbit Mechanics\\ Exercise 2: Numericial Integration of Satellite Orbits}
\rhead{Hsin-Feng Ho \\03770686}
\rfoot{Page \thepage}	
\begin{document}
\begin{titlepage}
    \vspace{\fill}
    \title{\textbf{Orbit Mechanics \\ Exercise 2: Numericial Integration of Satellite Orbits}}
    \vspace{5cm}
    \author{Hsin-Feng Ho\\
    03770686}
    \vspace{3cm}
    \maketitle
\end{titlepage}
\section*{Introduction}
In the first exercise we have seeen that the orbit of a satellite can be calculated by an analytical solution of the two-body problem. The satellite orbit can also be determined by numerical integration of the equations of motion.
\\\\
The Keplerian elements for the Sentinel-3 satellite is given:
\begin{table}[H]
\centering
\renewcommand\arraystretch{1.5}
\setlength{\tabcolsep}{5mm}{
\begin{tabular}{|c|c|c|c|c|c|c|}
    \hline
    \textbf{Satellite}& \textbf{a[km]}&\textbf{e}&\textbf{i[deg]}&\textbf{$\boldsymbol{\Omega}$[deg]}&\textbf{$\boldsymbol{\omega}$[deg]}&\textbf{$\boldsymbol{T_0}$[s]}\\\hline
    Sentinel-3&7192&0.004&98.3&257.7&144.2&00:00\\\hline
\end{tabular}
}
\end{table}
\section*{Undisturbed Orbit}
Using the given Keplerian elements and analytical solution of two body problem we can calculate the orbit of the Sentinel-3 satellite. 
\begin{figure}[H]
\centering
\includegraphics[width=0.8\textwidth]{./plots/3D_orbit.png}
\caption{undisturbed orbit of Sentinel-3 satellite for 3 revolutions}
\end{figure}
Now we have to calculate the satellite orbit using numerical integration. The 2nd order differential equation for the two-body problem is:
\begin{align*}
	\ddot{\textbf{r}}_i=-\frac{GM}{\left|\left|\textbf{r}_i\right|\right|^3}\textbf{r}_i
\end{align*}
where $G$ is the gravitational constant, $M$ is the mass of the Earth and $\textbf{r}_i$ is the position vector of the satellite. The equations of motion can be written as a system of first order differential equations:
\begin{align*}
	\begin{pmatrix}
	v_x\\v_y\\v_z
	\end{pmatrix}&=\begin{pmatrix}
	\dot{r_x}\\\dot{r_y}\\\dot{r_z}
	\end{pmatrix}\\
	\begin{pmatrix}
	\dot{v_x}\\\dot{v_y}\\\dot{v_z}
	\end{pmatrix}&=-\frac{GM}{R^3}\begin{pmatrix}
	r_x\\r_y\\r_z
	\end{pmatrix}
\end{align*}
where $R=\sqrt{r_x^2+r_y^2+r_z^2}$.
\\\\
Matlab offers several numerical integration functions. Now we want to compare the function \textbf{ode23} and \textbf{ode45} with step size of 5 and 50. The results are shown in the following figures:
\begin{figure}[H]
\centering
\begin{subfigure}[b]{0.45\textwidth}
\includegraphics[width=1\textwidth]{./plots/ode23_5_yprime.png}
\end{subfigure}
\begin{subfigure}[b]{0.45\textwidth}
\includegraphics[width=1\textwidth]{./plots/ode23_50_yprime.png}
\end{subfigure}
\begin{subfigure}[b]{0.45\textwidth}
\includegraphics[width=1\textwidth]{./plots/ode45_5_yprime.png}
\end{subfigure}
\begin{subfigure}[b]{0.45\textwidth}
\includegraphics[width=1\textwidth]{./plots/ode45_50_yprime.png}
\end{subfigure}
\end{figure}
By observing the figures we can see that \textbf{ode45} is more accurate than \textbf{ode23}. The smaller the step size is, the more accurate the result is. The result of \textbf{ode45} with step size of 5 can even reach a precision of $10^{-6}$m in position. In comparison, the result of \textbf{ode23} with step size of 5 is only accurate to meter level in position. With step size of 50, the result of \textbf{ode23} has a difference of 5km position and 5m/s velocity compared to the analytical solution, which we should avoid in practice.
\subsection*{Decomposition of the Error in RSW system}
The error of the numerical integration can be decomposed into radial, along-track and cross-track directions in the RSW system. The radial direction is the direction from the satellite to the center of the Earth. The along-track direction is the direction of the velocity vector of the satellite. The cross-track direction is the direction perpendicular to the radial and along-track directions. The error in the RSW system can be calculated by:
\begin{align*}
    \textbf{e}_R=\frac{\textbf{r}}{|\textbf{r}|}\qquad\textbf{e}_W=\frac{\textbf{r}\times\textbf{v}}{|\textbf{r}\times\textbf{v}|}\qquad\textbf{e}_S=\textbf{e}_W\times\textbf{e}_R\\
    \Delta r_R=\textbf{e}_R\cdot\Delta\textbf{r}\qquad\Delta r_W=\textbf{e}_W\cdot\Delta\textbf{r}\qquad\Delta r_S=\textbf{e}_S\cdot\Delta\textbf{r}
\end{align*}
The results are shown in the following figures:
\begin{figure}[H]
    \centering
    \begin{subfigure}[b]{0.45\textwidth}
    \includegraphics[width=1\textwidth]{./plots/ode23_5_yprime_RSW.png}
    \end{subfigure}
    \begin{subfigure}[b]{0.45\textwidth}
    \includegraphics[width=1\textwidth]{./plots/ode23_50_yprime_RSW.png}
    \end{subfigure}
    \begin{subfigure}[b]{0.45\textwidth}
    \includegraphics[width=1\textwidth]{./plots/ode45_5_yprime_RSW.png}
    \end{subfigure}
    \begin{subfigure}[b]{0.45\textwidth}
    \includegraphics[width=1\textwidth]{./plots/ode45_50_yprime_RSW.png}
    \end{subfigure}
    \end{figure}
    We can see that the error is mostly in the radial direction. The error in the along-track and cross-track directions are relatively small. Compare the two different step sizes, we can see that the error in the radial direction is smaller with smaller step size. 
    \section*{Disturbed Orbit}
    Now we want to add the perturbation of the Earth's oblateness to the equations of motion. The perturbation of the Earth's oblateness is given by:
    \begin{equation*}
        \ddot{\textbf{r}}_i=-\frac{GM}{r^3}\textbf{r}_i+\frac{3}{2}\frac{J_2a_e^2}{r^2}\begin{pmatrix}
            x\left(5\left(\frac{z}{r}\right)^2-1\right)\\
            y\left(5\left(\frac{z}{r}\right)^2-1\right)\\
            z\left(5\left(\frac{z}{r}\right)^2-3\right)
        \end{pmatrix}
    \end{equation*}
    where $J_2$ is the second zonal harmonic coefficient, $a_e$ is the equatorial radius of the Earth and $r=\sqrt{x^2+y^2+z^2}$.
    \\\\
    The results of the numerical integration with the perturbation of the Earth's oblateness are shown in the following figures:
    \begin{figure}[H]
        \centering
        \begin{subfigure}[b]{0.45\textwidth}
        \includegraphics[width=1\textwidth]{./plots/ode23_5_yprime_d.png}
        \end{subfigure}
        \begin{subfigure}[b]{0.45\textwidth}
        \includegraphics[width=1\textwidth]{./plots/ode23_50_yprime_d.png}
        \end{subfigure}
        \begin{subfigure}[b]{0.45\textwidth}
        \includegraphics[width=1\textwidth]{./plots/ode45_5_yprime_d.png}
        \end{subfigure}
        \begin{subfigure}[b]{0.45\textwidth}
        \includegraphics[width=1\textwidth]{./plots/ode45_50_yprime_d.png}
        \end{subfigure}
    \end{figure}
    \subsection*{In RSW frame}
    \begin{figure}[H]
        \centering
        \begin{subfigure}[b]{0.45\textwidth}
        \includegraphics[width=1\textwidth]{./plots/ode23_5_yprime_d_RSW.png}
        \end{subfigure}
        \begin{subfigure}[b]{0.45\textwidth}
        \includegraphics[width=1\textwidth]{./plots/ode23_50_yprime_d_RSW.png}
        \end{subfigure}
        \begin{subfigure}[b]{0.45\textwidth}
        \includegraphics[width=1\textwidth]{./plots/ode45_5_yprime_d_RSW.png}
        \end{subfigure}
        \begin{subfigure}[b]{0.45\textwidth}
        \includegraphics[width=1\textwidth]{./plots/ode45_50_yprime_d_RSW.png}
        \end{subfigure}
    \end{figure} 
\section*{Own Implementation of the Numerical Integration}
Now we want to implement our own numerical integration function. A simple case is to use Euler method to solve the differential equation. The Euler method is given by:
\begin{align*}
    \textbf{y}_{n+1}=\textbf{y}_n+h\textbf{f}(t_n,\textbf{y}_n)
\end{align*}
A more precise method is the Runge-Kutta method. The Runge-Kutta 4th order method is given by:
\begin{align*}
    \textbf{y}_{n+1}&=\textbf{y}_n+\frac{h}{6}(k_1+2k_2+2k_3+k_4)\\
    k_1&=\textbf{f}(t_n,\textbf{y}_n)\\
    k_2&=\textbf{f}(t_n+\frac{h}{2},\textbf{y}_n+\frac{h}{2}k_1)\\
    k_3&=\textbf{f}(t_n+\frac{h}{2},\textbf{y}_n+\frac{h}{2}k_2)\\
    k_4&=\textbf{f}(t_n+h,\textbf{y}_n+hk_3)
\end{align*}
The results of the numerical integration with the Euler method and Runge-Kutta method are shown in the following figures:
\begin{figure}[H]
    \centering
    \begin{subfigure}[b]{0.45\textwidth}
    \includegraphics[width=1\textwidth]{./plots/euler_5_yprime.png}
    \end{subfigure}
    \begin{subfigure}[b]{0.45\textwidth}
    \includegraphics[width=1\textwidth]{./plots/euler_50_yprime.png}
    \end{subfigure}
    \begin{subfigure}[b]{0.45\textwidth}
    \includegraphics[width=1\textwidth]{./plots/rk4_5_yprime.png}
    \end{subfigure}
    \begin{subfigure}[b]{0.45\textwidth}
    \includegraphics[width=1\textwidth]{./plots/rk4_50_yprime.png}
    \end{subfigure}
\end{figure}

\end{document}